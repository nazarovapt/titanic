\documentclass[a4paper, amsfonts, amssymb, amsmath, reprint, showkeys, nofootinbib, twoside]{revtex4-1}
\usepackage[ngerman]{babel}
\usepackage[utf8]{inputenc}
\usepackage[colorinlistoftodos, color=green!40, prependcaption]{todonotes}
\input{preamble}
\usepackage[pdftex, pdftitle={Article}, pdfauthor={Author}]{hyperref} % For hyperlinks in the PDF
%\setlength{\marginparwidth}{2.5cm}
\bibliographystyle{apsrev4-1}
\begin{document}
\title{Kurzbericht: Analyse der Titanic-Daten}

\author{Meico Bastian Heil}
\author{Saskia Lapkowski}
\author{Heetae Kim}
\author{Leticia Kwanga}
\author{Sergei Nazarov}
    \affiliation{Technische Universität Dortmund}

\date{08.02.2026} % Leave empty to omit a date

\begin{abstract}
In dieser Arbeit wurde der Titanic-Datensatz im Rahmen einer Gruppenarbeit mithilfe von GitHub und R analysiert.  
Die Daten wurden zunächst bereinigt und sinnvoll transformiert, um fehlende Werte zu imputieren und neue Variablen zu erstellen.  
Anschließend wurden deskriptive und bivariate Analysen durchgeführt, um Faktoren zu identifizieren, die das Überleben der Passagiere beeinflussen, insbesondere Geschlecht, Passagierklasse und Ticketpreis.
\end{abstract}

\maketitle

\section{Einleitung} \label{sec:einleitung}
Ziel dieser Arbeit ist es, praktische Erfahrungen im gemeinsamen Arbeiten mit GitHub zu sammeln.  
Dazu werden in Gruppen einfache analytische Aufgaben an den Titanic-Daten bearbeitet,  
wobei LaTeX und R zur Dokumentation und Auswertung genutzt werden.   
\section{Daten} \label{sec:daten}

Die Daten stammen aus der Titanic-Datenbank und wurden für die Analyse bereinigt und vereinheitlicht. Die wichtigsten Variablen nach der Aufbereitung sind in Tabelle I zusammengefasst.
\begin{table}[h!]
\centering
\resizebox{\columnwidth}{!}{%
\begin{tabular}{lll}
\hline
\textbf{Spalte} & \textbf{Bedeutung / Inhalt} & \textbf{Format} \\
\hline
Survived & Überlebt? Ja (1) / Nein (0) & factor \\
Pclass & Passagierklasse (1 = 1., 2 = 2., 3 = 3.) & ordered factor \\
Sex & Geschlecht des Passagiers & factor \\
Age & Alter in Jahren & numeric \\
SibSp & Anzahl Geschwister / Ehepartner an Bord & numeric \\
Parch & Anzahl Eltern / Kinder an Bord & numeric \\
Fare & Ticketpreis & numeric \\
Embarked & Einschiffungshafen (C / Q / S) & factor \\
Anrede & Anrede / Titel aus dem Namen & factor \\
\hline
\end{tabular}%
}
\caption{Struktur der Titanic-Daten}
\label{tab:datenstruktur}
\end{table}

Nach der Bereinigung wurden fehlende Werte behandelt und die Variablen in geeignete Formate konvertiert, sodass sie für die Analyse in R genutzt werden können.

\section{Methoden}

Die Analyse der Titanic-Daten erfolgt im Rahmen einer explorativen
deskriptiven Auswertung:

\begin{enumerate}
    \item Deskriptive Beschreibung zentraler metrischer Variablen (Alter, Ticketpreis sowie Anzahl mitreisender Angehöriger).
    \item Deskriptive Analyse kategorialer Merkmale (Überlebensstatus, Geschlecht, Passagierklasse, Einschiffungshafen).
    \item Untersuchung des Zusammenhangs zwischen Überlebensstatus und ausgewählten kategorialen Variablen (Geschlecht, Passagierklasse).
    \item Vergleich der Verteilungen von Alter und Ticketpreis zwischen Überlebenden und Nicht-Überlebenden.
    \item Visuelle Analyse des Zusammenspiels von Passagierklasse, Einschiffungshafen, Überlebensstatus und Geschlecht.
\end{enumerate}

\section{Ergebnisse}

\subsection*{1. Deskriptive Statistik der metrischen Variablen}

Die deskriptive Analyse der metrischen Variablen ergibt:

\begin{center}
\begin{tabular}{lccccc}
Variable & Mittelwert & Median & SD & Min & Max \\
\hline
Age  & 29.37 & 30.00 & 13.25 & 0.42 & 80.00 \\
Ticketpreis & 32.20 & 14.45 & 49.69 & 0.00 & 512.33 \\
\end{tabular}
\end{center}

Das durchschnittliche Alter der Passagiere beträgt etwa 29 Jahre.

\begin{figure}[h]
    \begin{center}
        \includegraphics[width=0.5\textwidth]{fare.png}
    \end{center}
    \caption{Histogramm der Ticketpreise (Fare)}
    \label{fig:fare_histogram}
\end{figure}

Der Ticketpreis weist eine rechtsschiefe Verteilung auf. Der Mittelwert liegt bei 32,
der Median nur bei 14. 

\subsection*{2. Deskriptive Statistik der kategorialen Variablen}

\begin{itemize}
    \item \textbf{Überlebensstatus:} Die Mehrheit der Passagiere hat die Katastrophe nicht überlebt.

    \begin{center}
\begin{tabular}{lcc}
Kategorie & Anzahl & Anteil (\%) \\
\hline
Nicht überlebt & 549 & 61.6 \\
Überlebt       & 342 & 38.4 \\
\end{tabular}
\end{center}

    \item \textbf{Geschlecht:} Deutlich mehr männliche als weibliche Passagiere an Bord.
    
    \begin{center}
\begin{tabular}{lcc}
Geschlecht & Anzahl & Anteil (\%) \\
\hline
weiblich & 314 & 35.2 \\
männlich & 577 & 64.8 \\
\end{tabular}
\end{center}

    \item \textbf{Passagierklasse:} Die Mehrheit reiste in der dritten Klasse.
    
    \begin{center}
\begin{tabular}{lcc}
Klasse & Anzahl & Anteil (\%) \\
\hline
1. Klasse & 216 & 24.2 \\
2. Klasse & 184 & 20.7 \\
3. Klasse & 491 & 55.1 \\
\end{tabular}
\end{center}

    \item \textbf{Einschiffungshafen:} Die meisten Passagiere stiegen in Southampton ein.
    
\begin{center}
\begin{tabular}{lcc}
Hafen & Anzahl & Anteil (\%) \\
\hline
C & 168 & 18.9 \\
Q & 77  & 8.6 \\
S & 644 & 72.3 \\
\end{tabular}
\end{center}
\end{itemize}
Abb. 2 zeigt eine allgemeine Übersicht


\begin{figure}
    \centering
    \includegraphics[width=0.5\textwidth]{kombo1.png}
    \caption{Übersicht der Passagierkategorien}
\end{figure}

\subsection*{3. Bivariate deskriptive Statistik für zwei kategoriale Variablen}

\textbf{Überlebensstatus nach Geschlecht:} Unter den Überlebenden ist der Anteil der Frauen deutlich höher; unter den Nicht-Überlebenden überwiegen die Männer.

\begin{figure}[h]
    \centering
    \includegraphics[width=0.50\textwidth]{survived_sex_bar.png}
    \caption{Überlebensstatus nach Geschlecht (Anteil)}
    \label{fig:survived_sex}
\end{figure}
    
\textbf{Überlebensstatus nach Passagierklasse:} In der 1. Klasse überleben deutlich mehr Passagiere, während die 3. Klasse den höchsten Anteil Nicht-Überlebender aufweist.

\begin{figure}[h]
    \centering
    \includegraphics[width=0.50\textwidth]{survived_class_bar.png}
    \caption{Überlebensstatus nach Passagierklasse (Anteil)}
    \label{fig:survived_class}
\end{figure}
    
\textbf{Passagierklasse nach Einschiffungshafen:} Passagiere aus Cherbourg gehörten häufiger der 1. Klasse an, was auf einen höheren sozioökonomischen Status hindeutet (Abb. 5).

\begin{figure}[h]
    \centering
    \includegraphics[width=0.50\textwidth]{class_embarked_bar.png}
    \caption{Passagierklasse nach Einschiffungshafen}
    \label{fig:class_embarked}
\end{figure}


\textbf{Überlebensraten nach Anrede:}
\begin{center}
\begin{tabular}{lcc}
Anrede & Nicht überlebt & Überlebt \\
\hline
Mr.     & 436 & 81  \\
Miss.   & 55  & 127 \\
Mrs.    & 26  & 99  \\
Master. & 17  & 23  \\
Andere  & 18  & 12  \\
\end{tabular}
\end{center}

Die Anrede zeigt deutliche Unterschiede im Überleben: Miss und Mrs. sind mit höheren Überlebensraten verbunden, während die Mehrheit der Mr.-Passagiere nicht überlebte.

\subsection*{4. Bivariate deskriptive Statistik}
Unterscheidet sich das Alter und der Ticketpreis der Passagiere in Abhängigkeit vom Überlebensstatus?

\textbf{Alter:} Überlebende sind im Median etwas jünger als Nicht-Überlebende. 

\textbf{Ticketpreis:} Überlebende haben im Durchschnitt höhere Ticketpreise gezahlt


Dies deutet darauf hin, dass sowohl Alter als auch finanzieler Status einen Einfluss
auf die Überlebenswahrscheinlichkeit hatten.
\begin{figure}[h]
    \centering
    \includegraphics[width=0.5\textwidth]{age_fare_survived_boxes.png}
    \caption{Alter und Ticketpreis nach Überlebensstatus}
    \label{fig:age_survived}
\end{figure}

\subsection*{5. Visualisierung von mehreren kategorialen Variablen}

\textbf{Überlebensstatus nach Passagierklasse, Einschiffungshafen und Geschlecht:} Frauen weisen in allen Klassen und Häfen höhere Überlebensanteile auf. Der Einfluss der Passagierklasse bleibt weiterhin bestehen.

\begin{figure}[h]
    \centering
    \includegraphics[width=0.5\textwidth]{survived_pclass_embarked_sex.png}
    \caption{Überlebensstatus nach Passagierklasse, Ein-
schiffungshafen und Geschlecht}
    \label{fig:age_pclass}
\end{figure}

\subsection*{6.Weitere Analyse}
Unterscheidet sich das Alter der Passagiere zwischen den drei Klassen?

Passagiere der ersten Klasse sind im Median älter als die der dritten Klasse.
Die Streuung ist ähnlich, aber in der dritten Klasse gibt es einige sehr junge Passagiere.




\section{Fazit}

Alter, Geschlecht und Passagierklasse zeigen deutliche Zusammenhänge mit der Überlebenswahrscheinlichkeit. Frauen und Passagiere höherer Klassen überlebten häufiger, während das Alter einen moderaten Einfluss zeigt. Diese Faktoren sind damit die wichtigsten Prädiktoren für das Überleben auf der Titanic.

\end{document}
